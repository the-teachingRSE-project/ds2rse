% !TeX encoding = UTF-8
% !TeX program = pdflatex
% !BIB program = biber

\documentclass[
        english,biblatex
    ]{lni}

\addbibresource{my-paper.bib}


% Standard packages
\usepackage{graphicx}
\usepackage{longtable}
\usepackage{booktabs}
\usepackage{array}
\usepackage{multirow}
\usepackage{wrapfig}
\usepackage{float}
\usepackage{colortbl}
\usepackage{pdflscape}
\usepackage{tabularx}
\usepackage{threeparttable}
\usepackage{threeparttablex}
\usepackage[normalem]{ulem}
\usepackage{makecell}

\usepackage{framed} % Needed for code blocks

\usepackage{color}
\usepackage{fancyvrb}
\newcommand{\VerbBar}{|}
\newcommand{\VERB}{\Verb[commandchars=\\\{\}]}
\DefineVerbatimEnvironment{Highlighting}{Verbatim}{commandchars=\\\{\}}
% Add ',fontsize=\small' for more characters per line
\usepackage{framed}
\definecolor{shadecolor}{RGB}{241,243,245}
\newenvironment{Shaded}{\begin{snugshade}}{\end{snugshade}}
\newcommand{\AlertTok}[1]{\textcolor[rgb]{0.68,0.00,0.00}{#1}}
\newcommand{\AnnotationTok}[1]{\textcolor[rgb]{0.37,0.37,0.37}{#1}}
\newcommand{\AttributeTok}[1]{\textcolor[rgb]{0.40,0.45,0.13}{#1}}
\newcommand{\BaseNTok}[1]{\textcolor[rgb]{0.68,0.00,0.00}{#1}}
\newcommand{\BuiltInTok}[1]{\textcolor[rgb]{0.00,0.23,0.31}{#1}}
\newcommand{\CharTok}[1]{\textcolor[rgb]{0.13,0.47,0.30}{#1}}
\newcommand{\CommentTok}[1]{\textcolor[rgb]{0.37,0.37,0.37}{#1}}
\newcommand{\CommentVarTok}[1]{\textcolor[rgb]{0.37,0.37,0.37}{\textit{#1}}}
\newcommand{\ConstantTok}[1]{\textcolor[rgb]{0.56,0.35,0.01}{#1}}
\newcommand{\ControlFlowTok}[1]{\textcolor[rgb]{0.00,0.23,0.31}{\textbf{#1}}}
\newcommand{\DataTypeTok}[1]{\textcolor[rgb]{0.68,0.00,0.00}{#1}}
\newcommand{\DecValTok}[1]{\textcolor[rgb]{0.68,0.00,0.00}{#1}}
\newcommand{\DocumentationTok}[1]{\textcolor[rgb]{0.37,0.37,0.37}{\textit{#1}}}
\newcommand{\ErrorTok}[1]{\textcolor[rgb]{0.68,0.00,0.00}{#1}}
\newcommand{\ExtensionTok}[1]{\textcolor[rgb]{0.00,0.23,0.31}{#1}}
\newcommand{\FloatTok}[1]{\textcolor[rgb]{0.68,0.00,0.00}{#1}}
\newcommand{\FunctionTok}[1]{\textcolor[rgb]{0.28,0.35,0.67}{#1}}
\newcommand{\ImportTok}[1]{\textcolor[rgb]{0.00,0.46,0.62}{#1}}
\newcommand{\InformationTok}[1]{\textcolor[rgb]{0.37,0.37,0.37}{#1}}
\newcommand{\KeywordTok}[1]{\textcolor[rgb]{0.00,0.23,0.31}{\textbf{#1}}}
\newcommand{\NormalTok}[1]{\textcolor[rgb]{0.00,0.23,0.31}{#1}}
\newcommand{\OperatorTok}[1]{\textcolor[rgb]{0.37,0.37,0.37}{#1}}
\newcommand{\OtherTok}[1]{\textcolor[rgb]{0.00,0.23,0.31}{#1}}
\newcommand{\PreprocessorTok}[1]{\textcolor[rgb]{0.68,0.00,0.00}{#1}}
\newcommand{\RegionMarkerTok}[1]{\textcolor[rgb]{0.00,0.23,0.31}{#1}}
\newcommand{\SpecialCharTok}[1]{\textcolor[rgb]{0.37,0.37,0.37}{#1}}
\newcommand{\SpecialStringTok}[1]{\textcolor[rgb]{0.13,0.47,0.30}{#1}}
\newcommand{\StringTok}[1]{\textcolor[rgb]{0.13,0.47,0.30}{#1}}
\newcommand{\VariableTok}[1]{\textcolor[rgb]{0.07,0.07,0.07}{#1}}
\newcommand{\VerbatimStringTok}[1]{\textcolor[rgb]{0.13,0.47,0.30}{#1}}
\newcommand{\WarningTok}[1]{\textcolor[rgb]{0.37,0.37,0.37}{\textit{#1}}}

% solve \tightlist error
\providecommand{\tightlist}{%
    \setlength{\itemsep}{0pt}\setlength{\parskip}{0pt}}

% solve \pandocbounded error
\providecommand{\pandocbounded}[1]{#1}



\begin{document}

% Title
        \title[]{The difference between RSE and Data Science}
    
    
    \author[1]{Julian Dehne}{julian.dehne@gi.de}{0000-0001-9265-9619}
\author[2]{Jan Philipp Thiele}{jan-philipp.thiele@tu-braunschweig.de}{0000-0002-8901-6660}
\author[3]{Jeremy Cohen}{jeremy.cohen@imperial.ac.uk}{0000-0003-4312-2537}
\author[3]{Firstname4 Lastname4}{firstname4.lastname4@affiliation1.org}{0000-0000-0000-0000}
\affil[1]{Gesellschaft für Informatik\\Bildung und Gesellschaft\\
Weydingerstraße 14-16\\10178 Berlin\\Deutschland}
\affil[2]{TU Braunschweig\\Universitätsbibliothek\\
Universitätspl. 1, 38106 Braunschweig\\Deutschland}
\affil[3]{University 3\\Department\\Address\\Country}


    \maketitle

% Abstract
        \begin{abstract}
        Die LaTeX-Klasse \texttt{lni} setzt die Layout-Vorgaben für
        Beiträge in LNI Konferenzbänden um. Dieses Dokument beschreibt
        ihre Verwendung und ist ein Beispiel für die entsprechende
        Darstellung.
    \end{abstract}
    
% Keywords
        \begin{keywords}
        LNI Guidelines \and LaTeX Vorlage
    \end{keywords}
    
% Body
    \section{Verwendung}\label{verwendung}

    Die GI gibt unter http://www.gi-ev.de/LNI Vorgaben für die
    Formatierung von Dokumenten in der LNI Reihe.

    \section{Demonstrationen}\label{demos}

    \section{Demonstration der Einhaltung der
    Richtlinien}\label{lniconformance}

    \subsection{Literaturverzeichnis}\label{literaturverzeichnis}

    Beispielhafte Zitierungen: \textcite{Ez10}, \textcite{AB00}.

    \section{Abbildungen}\label{abbildungen}

    \begin{figure}

    \centering{

    \includegraphics[width=0.8\linewidth,height=\textheight,keepaspectratio]{four_pillars_rse.png}

    }

    \caption{\label{fig-demo}Demographik}

    \end{figure}%

    \section{Tabellen}\label{tabellen}

    \begin{longtable}[]{@{}lll@{}}
    \caption{Die Überschriftsarten \{\#tab-demo\}}\tabularnewline
    \toprule\noalign{}
    Überschriftsebenen & Beispiel & Schriftgröße und -art \\
    \midrule\noalign{}
    \endfirsthead
    \toprule\noalign{}
    Überschriftsebenen & Beispiel & Schriftgröße und -art \\
    \midrule\noalign{}
    \endhead
    \bottomrule\noalign{}
    \endlastfoot
    Titel (linksbündig) & Der Titel \ldots{} & 14 pt, Fett \\
    Überschrift 1 & 1 Einleitung & 12 pt, Fett \\
    Überschrift 2 & 2.1 Titel & 10 pt, Fett \\
    \end{longtable}

    \section{Programmcode}\label{programmcode}

\begin{Shaded}
\begin{Highlighting}[]
\KeywordTok{public} \KeywordTok{class}\NormalTok{ Hello }\OperatorTok{\{}
    \KeywordTok{public} \DataTypeTok{static} \DataTypeTok{void} \FunctionTok{main} \OperatorTok{(}\BuiltInTok{String}\OperatorTok{[]}\NormalTok{ args}\OperatorTok{)} \OperatorTok{\{}
        \BuiltInTok{System}\OperatorTok{.}\FunctionTok{out}\OperatorTok{.}\FunctionTok{println}\OperatorTok{(}\StringTok{"Hello World!"}\OperatorTok{);}
    \OperatorTok{\}}
\OperatorTok{\}}
\end{Highlighting}
\end{Shaded}

    \section{Formeln}\label{formeln}

    \section{RSE and DS embeddings in the Research
    Cycle}\label{rse-and-ds-embeddings-in-the-research-cycle}

    Both RSE and DS can be conceptualized as a cross-cutting concern in
    many disciplines. However, the definition and relevance of these
    issues can be generalized based on the function they fulfill in the
    research cycle Figure~\ref{fig-research_cycle}.

    \newpage

    \begin{figure}

    \centering{

    \includegraphics[width=0.7\linewidth,height=\textheight,keepaspectratio]{img/research_cycle_wildt.png}

    }

    \caption{\label{fig-research_cycle}The research cycle
    \autocite{wildt2009forschendes} integrates the typical research
    process with the learning process.}

    \end{figure}%

    There are different research processes depending on the discipline
    and the research question. However, \autocite{Dehne2021} showed that
    most of the research processes contain the following phases:

    \begin{enumerate}
    \def\labelenumi{\arabic{enumi}.}
    \tightlist
    \item
      conceptualization (developing research questions, concepts)
    \item
      design (developing the tools, instruments and concrete process
      models)
    \item
      implementation (executing the experiment, study)
    \item
      analysis \& interpretation
    \item
      dissemination (publishing, distributing, peer-review)
    \item
      reflexion and improvements
    \end{enumerate}

    For example, in the case of the field of learning technologies, the
    design phase often consists of extensive software development of
    different tools for learning. In this situation developing complex
    tools for learning can be considered as research software
    engineering. The analysis of what learners can gain from using these
    technologies can be conceived as an educational research in its own
    right. This example highlights the differences of scale in both the
    weight of the different process phases (here: phase 2) and the
    relevance of research engineering. It also shows a situation where
    research software engineering clearly differs from data science. A
    typical data science background would not enable researchers to
    build full-stack software that solves inefficiencies or
    hard-to-teach problems in education.

    In the second example a GPT-like attention model is trained to
    classify data gained from the James-Webb telescope. Due to vast
    amounts of data and the continuous stream of new data research
    software engineering is needed to implement a pipeline for data
    cleaning, data warehousing and in-time analysis. In this case, the
    analysis \& interpretation phase (4) has much more relevance.
    Another point of this is example is that data science competencies
    such as vectorization of algorithms, statistical analysis etc. are
    interconnected with competencies from software engineering such as
    software architectures, software project management, and database
    programming.

    The main argument behind these examples is that data science and
    research software engineering have a lot in common in terms of
    software development for science but show major differences
    depending on the research process in question.

    \section{What has been discussed}\label{what-has-been-discussed}

    TODO insert: Levels of Differences between RSE and DS:

    \begin{itemize}
    \tightlist
    \item
      institutional
    \item
      disciplinary connections
    \item
      target groups
    \item
      political history
    \end{itemize}

    \section{DS and RSE Competences in relation to the Research
    Cycle}\label{ds-and-rse-competences-in-relation-to-the-research-cycle}

    \section{Case Study}\label{case-study}

    How much RSE and DS

    \section{Discussion}\label{discussion}

    TODO: fill What is the difference between the research phases,
    competences and in general between DS and RSE Should there be a
    third comparison to scientific computing?

    Alignment vs.~identiy between DS and RSE

% Bibliography

    \printbibliography

\end{document}
